\documentclass[twoside, 11pt]{myreport}

%% ----------------------- %%
%% start of the 'preamble' %%
%% ----------------------- %%

%___________________________________________________________
%
% -- Declaration of packages
\usepackage{amsmath} % Maths pour avoir
\usepackage{amssymb} % les ensembles

\usepackage{epsfig} % Pour afficher des figures
\usepackage{subfig} % Pour afficher plusieurs figures cote a cote
\usepackage{wrapfig} % Pour pouvoir ecrire autour des figures

\usepackage{color}	% Pour utiliser les couleurs
\definecolor{red}{rgb}{1,0,0}	% Definition de la couleur rouge
\definecolor{darkblue}{rgb}{0,0,0.2}

% \usepackage[T1]{fontenc}    % Carateres accentues
% \usepackage[frenchb]{babel} % Typographie francaise

\usepackage[top=3cm, bottom=3cm, left=2.5cm, right=2.5cm]{geometry}

\usepackage{fancyhdr} % Pour modifier les en-tete et pied de page
\usepackage{lastpage} % Pour avoir acces au numero de la derniere page

\usepackage{hyperref} % Pour avoir des liens vers les differentes parties
\hypersetup{
		colorlinks = true, % Pour virer les rectangles rouges autour des liens
		linkcolor = darkblue, %black,
		citecolor = darkblue, %black,
		filecolor = darkblue, %black,
		urlcolor = darkblue, %black
}

% -- End of declaration of packages
%___________________________________________________________
%

%___________________________________________________________
%
% -- Various expressions
\newcommand{\todo}[1]{\textbf{\emph{\textcolor{red}{#1}}}} % Definition de la commande todo (met le texte en rouge)
\newcommand{\refeq}[1]{(\ref{#1})} % Pour mettre des parentheses autour des numeros d'equations

\newcommand{\bs}{B-spline}
\newcommand{\eg}{\emph{e.g. }}
\newcommand{\etal}{\emph{et al. }}
\newcommand{\ie}{\emph{i.e. }}
\newcommand{\ls}{level-set}
\newcommand{\Ls}{Level-set}
\newcommand{\LS}{$\phi(\cdot)$}

\newcommand{\vecx}{\mathbf}

\def\Rset{\mathbb{R}}
\def\Zset{\mathbb{Z}}
\def\Nset{\mathbb{N}}
%___________________________________________________________
%
% -- End of various expressions

%%___________________________________________________________
%%
%% -- Definition of headers and footers
\pagestyle{fancy}
\renewcommand{\chaptermark}[1]{\markboth{\thechapter .~#1}{}}

\fancyhead{} % Headers
\fancyhead[LE,RO]{\nouppercase\leftmark}
\fancyhead[LO,RE]{\nouppercase\rightmark}

\fancyfoot{} % Footers
\fancyfoot[CE,CO]{\thepage /\pageref{LastPage}}

\headsep=20pt % Definition of spaces between header and text
\headheight=16pt % Header's size
\renewcommand{\footrulewidth}{0.5pt} % Width of the line separating the header from the text

%% -- Fin de la definition
%%___________________________________________________________
%%

%% ----------------- %%
%% end of 'preamble' %%
%% ----------------- %%

%% -------------- %%
%% begin document %%
%% -------------- %%
\begin{document}
%___________________________________________________________
%
% -- Declaration of each part

% ---------- %
% Title Page %
% ---------- %

\begin{titlepage}
 
\begin{center}
 
 
% Upper part of the page
\vspace*{0.5cm}
\vfill
 
% Title
\hrule 
\vspace*{0.4cm}
\Huge \bfseries {\Ls}
\vspace*{0.4cm} 
\hrule 

\vfill

\end{center}
 
\end{titlepage}

% -------- %
% Sommaire %
% -------- %

\newpage

\pagestyle{fancy}
\fancyhead{} % Headers
\fancyhead[LE,RO]{\nouppercase\leftmark}

\tableofcontents
\addcontentsline{toc}{chapter}{Table of Contents}


\newpage

\pagestyle{fancy}
\fancyhead{} % Headers
\fancyhead[LE,RO]{Introduction}

\chapter*{Introduction}
\label{chap:intro}
\addcontentsline{toc}{chapter}{Introduction}

\vspace{0.5cm}
\hspace{0.7cm} This document gives a short description of some \ls-based algorithms. For further details about these methods, the reader is invited to refer to the cited algorithms. For each method, the energy criterion which is minimized, the derived evolution equation and the main properties of the method are given.
% ---------------- %
% Level-set Method %
% ---------------- %

\newpage

\fancyhead{} % Headers
\fancyhead[LE,RO]{\nouppercase\leftmark}
\fancyhead[LO,RE]{\nouppercase\rightmark}

\chapter{``Classical'' \ls~methods}
\label{chap:ls-methods}

\section[Caselles]{Caselles \cite{Caselles1997}}
\label{sec:Caselles}

\paragraph{Energy criterion}
\begin{equation}
	\label{eq:NRJ_caselles}
	E(\Gamma) = \int_0^1 g(I(\Gamma(q))) \|\Gamma'(q)\| dq,
\end{equation}
where
\begin{equation}
	\label{eq:g_caselles}
	g(I) = \frac{1}{1 + \| \nabla (G \ast I) \|^2},
\end{equation}
$I(\cdot)$ corresponds to the image intensity, $\Gamma$ is the parametric curve and $G$ is a gaussian filter of variance 1.

\paragraph{Evolution equation}
\begin{equation}
	\label{eq:evol_caselles}
	\frac{\partial \phi}{\partial t}(\vecx{x}) = g(I(\vecx{x})) \| \nabla \phi(\vecx{x}) \| \kappa + \nabla g(I(\vecx{x})) \nabla \phi(\vecx{x}).
\end{equation}
where $\kappa = \text{div}\left(\frac{\nabla \phi(\vecx{x})}{\| \nabla \phi(\vecx{x}) \|} \right)$ corresponds to the curvature of the evolving contour.

\paragraph{Properties}

\begin{itemize}
	\item This algorithm is a contour-based method \ie the gradient of the image is used to compute the force function. The curve will thus be driven to regions with high gradient.
	\item This method does not require any regularization term as it is intrinsic to the method.
\end{itemize}


\newpage
\section[Chan \& Vese]{Chan \& Vese \cite{ChanVese2001}}
\label{sec:ChanVese}

\paragraph{Energy criterion}
\begin{equation}
	\label{eq:NRJ_chanvese}
	E(\phi) = \int_\Omega F(I(\vecx{x}), \phi(\vecx{x})) \, d\vecx{x} + \lambda \int_\Omega \delta(\phi(\vecx{x})) \| \nabla \phi(\vecx{x}) \|d\vecx{x},
\end{equation}
where $\delta$ is the dirac function and 
\begin{equation}
	\label{eq:F_chanvese}
	F(I(\vecx{x}),\phi(\vecx{x})) = H(\phi(\vecx{x}))(I(\vecx{x}) - v)^2 + (1 - H(\phi(\vecx{x})))(I(\vecx{x}) - u)^2,
\end{equation}
$H$ is the Heaviside function, $u$ and $v$ are two parameters updated at each iteration as follows:

\begin{equation}
	\label{eq:u_chanvese}
	u = \frac{\int_\Omega (1 - H(\phi(\vecx{x}))) \cdot I(\vecx{x})\, d\vecx{x}}{\int_\Omega 1 - H(\phi(\vecx{x})) \, d\vecx{x}} 
\end{equation}
\begin{equation}
	\label{eq:v_chanvese}
	v = \frac{\int_\Omega H(\phi(\vecx{x})) \cdot I(\vecx{x})\, d\vecx{x}}{\int_\Omega H(\phi(\vecx{x})) \, d\vecx{x}} 
\end{equation}

The first integral of \refeq{eq:NRJ_chanvese} correspond to a data attached term and the second is a regularization term that acts on the evolving contour.

\paragraph{Evolution equation}
\begin{equation}
	\label{eq:evol_chanvese}
	\frac{\partial \phi}{\partial t}(\vecx{x}) = \delta(\phi(\vecx{x})) \nabla_\phi F(I(\vecx{x}), \phi(\vecx{x})) + \lambda \delta(\phi(\vecx{x})) \text{div}\left(\frac{\nabla \phi(\vecx{x})}{\|\nabla \phi(\vecx{x})\|}\right),
\end{equation}
where
\begin{equation}
	\label{eq:gradF_chanvese}
	\nabla_\phi F(I(\vecx{x}),\phi(\vecx{x})) = \delta(\phi(\vecx{x}))((I(\vecx{x}) - v)^2 - (I(\vecx{x}) - u)^2)
\end{equation}

\paragraph{Properties}

\begin{itemize}
	\item This algorithm is a region-based method. It tends to separate the image into two homogeneous region (according to their mean value).
	\item The evolution is only computed on the narrow-band of the \ls~thus making it sensitive to initialization.
\end{itemize}


\newpage
\section[Chunming Li]{Chunming Li \cite{Li2008}}
\label{sec:Li}

\paragraph{Energy criterion}
\begin{eqnarray}
	\label{eq:NRJ_li}
	\nonumber E(\phi) = & \lambda_1 \int \int K_\sigma(\vecx{x} - \vecx{y}) | I(\vecx{y}) - f_1(\vecx{x}) |^2 H(\phi(\vecx{x})) \, d\vecx{y}d\vecx{x} \\
	& + \lambda_2 \int \int K_\sigma(\vecx{x} - \vecx{y}) | I(\vecx{y}) - f_2(\vecx{x}) |^2 (1 - H(\phi(\vecx{x})))\, d\vecx{y}d\vecx{x} \\ 
	\nonumber & + \nu \int| \nabla H(\phi(\vecx{x})) |d\vecx{x} + \mu \int \frac{1}{2}\left(\|\nabla \phi(\vecx{x}) \| - 1\right)d\vecx{x},
\end{eqnarray}
where $I(\vecx{x})$ is the image intensity at pixel $\vecx{x}$, $H$ is the Heaviside function, $K_\sigma$ is a gaussian kernel defined as:
\begin{equation}
	\label{eq:Ksigma_li}
	K_\sigma(\vecx{u}) = \frac{1}{(2\pi)^{n/2}\sigma^n}e^{-\|\vecx{u}\|^2/2\sigma^2},
\end{equation}
with a scale parameter $\sigma > 0$. $f_1$ and $f_2$ are two functions centered at pixel $\vecx{x}$ and defined as:
\begin{equation}
	\label{eq:f1_li}
	f_1(\vecx{x}) = \frac{K_\sigma \ast (H(\phi(\vecx{x}))I(\vecx{x}))}{K_\sigma \ast H(\phi(\vecx{x}))},
\end{equation}
\begin{equation}
	\label{eq:f2_li}
	f_2(\vecx{x}) = \frac{K_\sigma \ast ((1 - H(\phi(\vecx{x})))I(\vecx{x}))}{K_\sigma \ast (1- H(\phi(\vecx{x})))}.
\end{equation}

The two first integrals of \refeq{eq:NRJ_li} correspond to data attached term. The third integral is a regularization term that minimizes the curve length. The last integral is a regularization term that forces the \ls~to keep signed distance properties over the evolution process.

\paragraph{Evolution equation}
\begin{eqnarray}
	\label{eq:evol_li}
	\nonumber \frac{\partial \phi}{\partial t}(\vecx{x}) = & \delta(\phi(\vecx{x})) \left( \lambda_1 \int K_\sigma(\vecx{x} - \vecx{y}) | I(\vecx{y}) - f_1(\vecx{x}) |^2 \, d\vecx{y} + \lambda_2 \int \int K_\sigma(\vecx{x} - \vecx{y}) | I(\vecx{y}) - f_2(\vecx{x}) |^2 \, d\vecx{y} \right) \\ 
	& + \nu \delta(\phi(\vecx{x})) \text{div}\left(\frac{\nabla \phi(\vecx{x})}{\|\nabla \phi(\vecx{x})\|}\right) + \mu \left( \nabla^2 \phi(\vecx{x}) - \text{div}\left(\frac{\nabla \phi(\vecx{x})}{\|\nabla \phi(\vecx{x})\|}\right) \right),
\end{eqnarray}

\paragraph{Properties}

\begin{itemize}
	\item Because of the localization effects introduced by $f_1$, $f_2$ and $K_\sigma$, this algorithm is able to segment inhomogeneous objects.
	\item This algorithm segments the whole image.
\end{itemize}


\newpage
\section[Lankton]{Lankton \cite{Lankton2008}}
\label{sec:Lankton}

\paragraph{Energy criterion}
\begin{equation}
	\label{eq:NRJ_lankton}
	E(\phi) = \int_{\Omega_x} \delta(\phi(\vecx{x})) \int_{\Omega_y} B(\vecx{x},\vecx{y}) \cdot F(I(\vecx{y}), \phi(\vecx{y})) \, d\vecx{y}d\vecx{x} + \lambda \int_{\Omega_x} \delta(\phi(\vecx{x})) \| \nabla \phi(\vecx{x}) \|d\vecx{x},
\end{equation}
where $\delta$ is the Dirac function, $B$ is a ball of radius $r$ centered at point $\vecx{x}$ and defined as follow:
\begin{equation}
	\label{eq:B}
	B(\vecx{x},\vecx{y}) = 
	\begin{cases}
		1, & \|\vecx{x}-\vecx{y}\| \leq r \\
		0, & otherwise,
	\end{cases}
\end{equation}
and
\begin{equation}
	\label{eq:F_lankton}
	F(I(\vecx{y}),\phi(\vecx{y})) = 
	\begin{cases}
		H(\phi(\vecx{y}))(I(\vecx{y}) - v_\vecx{x})^2 + (1 - H(\phi\vecx{y})))(I(\vecx{y}) - u_\vecx{x})^2, & \text{Chan \& Vese feature,}\\
		(v_\vecx{x} - u_\vecx{x})^2, & \text{Yezzi feature,}
	\end{cases}
\end{equation}
where $H$ is the Heaviside function, $u_\vecx{x}$ and $v_\vecx{x}$ are two parameters updated at each iteration as follows:

\begin{equation}
	\label{eq:u_lankton}
	u_\vecx{x} = \frac{\int_{\Omega_y} B\vecx{x},\vecx{y}) \cdot (1 - H(\phi(\vecx{y}))) \cdot \vecx{y})\, d\vecx{y}}{\int_{\Omega_y} B(\vecx{x},\vecx{y}) \cdot (1 - H(\phi(\vecx{y}))) \, d\vecx{y}} 
\end{equation}
\begin{equation}
	\label{eq:v_lankton}
	v_\vecx{x} = \frac{\int_{\Omega_y} B(\vecx{x},\vecx{y}) \cdot H(\phi(\vecx{y})) \cdot I(\vecx{y})\, d\vecx{y}}{\int_{\Omega_y} B(\vecx{x},\vecx{y}) \cdot H(\phi(\vecx{y})) \, d\vecx{y}} 
\end{equation}

The first integral of \refeq{eq:NRJ_lankton} correspond to a data attached term and the second is a regularization term that acts on the evolving contour.

\paragraph{Evolution equation}
\begin{equation}
	\label{eq:evol_lankton}
	\frac{\partial \phi}{\partial t}(\vecx{x}) = \delta(\phi(\vecx{x})) \int_{\Omega_y} B(\vecx{x},\vecx{y}) \cdot \nabla_\phi F(I(\vecx{y}), \phi(\vecx{y})) \, d\vecx{y} + \lambda \delta(\phi(\vecx{x})) \text{div}\left(\frac{\nabla \phi(\vecx{x})}{\|\nabla \phi(\vecx{x})\|}\right),
\end{equation}
where
\begin{equation}
	\label{eq:gradF_lankton}
	\nabla_\phi F(I(\vecx{y}),\phi(\vecx{y})) = 
	\begin{cases}
		\delta(\phi(\vecx{y}))((I(\vecx{y}) - v_\vecx{x})^2 - (I(\vecx{y}) - u_\vecx{x})^2), & \text{Chan \& Vese feature,}\\
		\delta(\phi(\vecx{y})) \left( \frac{(I(\vecx{y}) - v_\vecx{x})^2}{A_v} - \frac{(I(\vecx{y}) - u_\vecx{x})^2}{A_u}\right), & \text{Yezzi feature,}
	\end{cases}
\end{equation}
where $A_u$ and $A_v$ are the area of the local interior and local exterior regions respectively given by
\begin{equation}
	\label{eq:Au_lankton}
	A_u = \int_{\Omega_y} B(\vecx{x},\vecx{y}) \cdot (1 - H(\phi(\vecx{x}))) \, d\vecx{y}
\end{equation}
\begin{equation}
	\label{eq:Av_lankton}
	A_v = \int_{\Omega_y} B(\vecx{x},\vecx{y}) \cdot H(\phi(\vecx{x})) \, d\vecx{y}
\end{equation}

\paragraph{Properties}

\begin{itemize}
	\item This algorithm is a region-based method.
	\item Its feature term is computed locally. This property allows the algorithm to segment non homogeneous objects. However this make the method sensitive to initialization.
\end{itemize}


\newpage
\section[Bernard]{Bernard \cite{Bernard2009a}}
\label{sec:Bernard}

\paragraph{Model}
~\par \vspace{0.3cm}

Let $\Omega$ be a bounded open subset of $\Rset^{d}$ and let $f:\Omega\mapsto\Rset$ be a given $d$-dimensional image. In the \bs~\ls~ formalism, the evolving interface $\Gamma\subset\Rset^{d}$ is represented as the zero level-set of an implicit function \LS~expressed as a linear combination of \bs~ basis functions
\begin{equation}
\label{eq:phi-bernard}
\phi({\vecx{x}})=\sum_{{\vecx{k}}\in{\Zset}^{d}}\,c[{\vecx{k}}]\,\beta^{n} \left(\frac{{\vecx{x}}}{h}-{\vecx{k}}\right).
\end{equation}

Here, $\beta^{n}(\cdot)$ is the uniform symmetric $d$-dimensional \bs~ of degree $n$. The knots of the \bs~ are located on a grid spanning $\Omega$, with a regular spacing. The coefficients of the \bs~ representation are gathered in $c[\vecx{k}]$. $h$ is a scale parameter which directly influence the degree of smoothing of the interface.

\paragraph{Energy criterion}
\begin{equation}
	\label{eq:NRJ-bernard}
	E(\phi) = \int_{\Omega} F(I(\vecx{x}),\phi(\vecx{x})) \, d\vecx{x},
\end{equation}
where 
\begin{equation}
	\label{eq:F_bernard}
	F(I(\vecx{x}),\phi(\vecx{x})) = H(\phi(\vecx{x}))(I(\vecx{x}) - v)^2 + (1 - H(\phi(\vecx{x})))(I(\vecx{x}) - u)^2,
\end{equation}
$H$ is the Heaviside function, $u$ and $v$ are two parameters updated at each iteration according to equation \refeq{eq:u_chanvese} and \refeq{eq:v_chanvese}.

\paragraph{Evolution equation}
~\par \vspace{0.3cm}

The minimization of the functional \refeq{eq:NRJ-bernard} can be done with respect to the \bs~coefficients $c[\vecx{k}]$. The derivatives with respect to each \bs~coefficient $c[\vecx{k}_{\textbf{0}}]$ may be expressed as
\begin{equation}
	\label{eq:dEdck-bernard}
	\frac{\partial E}{\partial c[\vecx{k}_{\vecx{0}}]} = \int_{\Omega}{\frac{\partial F(\vecx{x},\phi(\vecx{x}))}{\partial \phi(\vecx{x})} \cdot \beta^{n}\left( \frac{\vecx{x}}{h} - \vecx{k}_{\vecx{0}} \right) \, d\vecx{x}},
\end{equation}
with 
\begin{equation}
	\label{eq:gradF-bernard}
	\frac{\partial F(\vecx{x},\phi(\vecx{x}))}{\partial \phi(\vecx{x})} = \delta(\phi(\vecx{x})( (I(\vecx{x})-v)^2 - (I(\vecx{x}) - u)^2 ).
\end{equation}

The level-set evolution may then be computed through a gradient descent on the \bs~coefficients. The corresponding variation of the \bs~coefficients is given as:
\begin{equation}
	\label{eq:evol-bernard}
	\vecx{c}^{i+1} = \vecx{c}^{i} - \lambda\nabla_{c}E(\vecx{c}^{i}),
\end{equation}
where $\lambda$ is the iteration step and $\nabla_{c}$ correspond to the gradient of the energy relative to the \bs~coefficients given by \refeq{eq:dEdck-bernard}.

\paragraph{Properties}

\begin{itemize}
	\item This algorithm computes the \ls~evolution on the whole image. So new contours could emerge far from the initialization.
	\item This algorithm is a region-based method and tries to separate the image into two homogeneous region (according to their means value).
\end{itemize}


\newpage
\section[Shi]{Shi \cite{Shi2008}}
\label{sec:Shi}

\paragraph{Model}
~\par \vspace{0.3cm}

\hspace{0.7cm} This method is a fast algorithm that approximate \ls~based curve evolution. The implicit function is represented using a limited set of integers (-3, -1, 1, 3) to define the interior points, the interior points adjacent to the evolving curve, the exterior points adjacent to the evolving curve, the exterior points. Moreover the points adjacent to the evolving curve are gathered into two lists: $L_{in}$ and $L_{out}$.
\vspace{0.3cm}

The curve evolution process is then approximated in a two-cycle algorithm:
\begin{enumerate}
	\item during Na iterations, the curve evolve using a data attachment term $F_d$.
	\item then the curve is smoothed during Ns iterations; the regularization term $F_r$ is computed for each point of $L_{in}$ and $L_{out}$ using a gaussian filter of variance $\sigma$ and size Ng$\times$Ng.
\end{enumerate}

\paragraph{Evolution equation}
~\par \vspace{0.3cm}
In \cite{Shi2008}, several data attached term are given. In the platform, we choose to use the Chan \& Vese one given by:
\begin{equation}
	\label{eq:F_shi}
	F(I(\vecx{x}),\phi(\vecx{x})) = H(\phi(\vecx{x}))(I(\vecx{x}) - v)^2 + (1 - H(\phi(\vecx{x})))(I(\vecx{x}) - u)^2
\end{equation}
where $H$ is the Heaviside function, $u$ and $v$ are two parameters updated at each iteration according to equation \refeq{eq:u_chanvese} and \refeq{eq:v_chanvese}.

\paragraph{Properties}

\begin{itemize}
	\item This method is an approximation of \ls~based curve evolution. It is a very fast method and evolves only on the narrow-band.
\end{itemize}

\input{src/shape_prior}
% --------------------- %
% A priori de mouvement %
% ---------------------%

\newpage
\chapter{\Ls~with motion prior}
\label{chap:mvt}

\section[Zhang and Pless]{Zhang and Pless 2005 \cite{Zhang2005}}
\label{sec:motion-zhang}

This method is designed for cardiopulmonary sequences segmentation. Thus the following framework only 2 degrees of freedom are used to parameterize the manifold.

\paragraph{Manifolds learning using isomap embedding}
~\par \vspace{0.3cm}
The Isomap procedure for dimensionality reduction starts by computing the distance between all pairs of images (using some distance function such as SSD pixel intensities). Then, a graph is defined with each image as a node and undirected edges connecting each image to its $k$-closest neighbors (usually choosing $k$ between $5$ and $10$). A complete pair-wise distance matrix is calculated by solving for the all-pairs shortest paths in this sparse graph. Finally, this complete distance matrix is embedded into some low dimension by solving an Eigenvalue problem (Multidimensional Scaling (MDS)). The dimensionality embedding can be chosen as desired, but ideally is the number of degrees of freedom in the image set.

For data sets with deformable motion, a suggested distance function is computed as the phase difference of local complex Gabor filters:
\begin{equation}
  \label{eq:local_Gabor_filter_zhang}
  \|I_1 - I_2 \|_{motion} = \sum_{x,y} { \psi(G(\omega, V, \sigma) \otimes I_1, G(\omega, V, \sigma) \otimes I_2) + \psi(G(\omega, H, \sigma) \otimes I_1, G(\omega, H, \sigma) \otimes I_2) }
\end{equation}
where $G(\omega, H, \sigma)$ is defined to be the 2D complex Gabor filter with frequency $\omega$, oriented either vertically or horizontally, with $\sigma$ as the variance of the modulating Gaussian, and $\psi$ returns the phase difference of the pair of complex Gabor responses above some threshold $\tau$.


\paragraph{Energy criterion}
~\par \vspace{0.3cm}
For cardiopulmonary image sequences, the images vary in principle depending on their cardiac phase $u$ and pulmonary phase $v$ - the two degrees of freedom that parameterize the manifold. Isomap is used to automatically parameterize all images, and interpolate the result to generate evenly spaced samples of the image manifold $f(x, y, u, v)$. Accordingly, the seeked contour $C$ is also a function of $u$ and $v$, and $C$ has to be described implicitly by the \ls~function $\phi$ in 4-dimension space $\Omega$. Thus, a given cardiopulmonary image sequence specifies this contour by extending the energy functional \refeq{eq:NRJ_chanvese} to 4-dimension space $\underset{c1,c2,\phi}{inf}E(c1,c2,\phi)$, where $c_1$ and $c_2$ are the interior and exterior means respectively and $\phi: \Rset^4 \rightarrow \Rset$.

But the manifold dimensions also correspond to specific kinds of deformation. The breathing of the patient results, approximately, in a translation of the heart. Therefore, the variation of $\phi$ in the $v$ direction is expected to be a uniform translation. That is, the energy functional change $\frac{\partial \phi}{\partial v}$ should be consistent with a uniform translation. This induces a \ls~corollary to the classic optic flow constraint equation:
\begin{equation}
  \label{eq:ls_OpticalFlow_zhang}
  \frac{\partial \phi}{\partial x}\omega_x + \frac{\partial \phi}{\partial y}\omega_y + \frac{\partial \phi}{\partial v} = 0
\end{equation}
where $(\omega_x, \omega_y)^T$ is the velocity vector that is constant over any given image, but may vary for different values of $u$ and $v$.

On the other hand, varying images along the other axis of the image manifold, deformations due to the cardiac cycle lead to image variation with minimal overall translation. For the special case of deformation caused by (non-uniform) heart expansion and contraction, the constraint can be expressed as:
\begin{equation}
  \label{eq:ls_dphidu_zhang}
  \frac{\partial \phi}{\partial u} = \omega_u
\end{equation}
where $\omega_u$ is constant over the region of the heart for any given $u$ and $v$. This constraint enforces the condition that moving along the ``heartbeat'' axis simply adds or subtracts a constant value of the \ls~function $\phi$, and therefore enforces that the shape either expands or shrinks.

Using these two constraints, the motion constraints can be written as an energy functional:
\begin{equation}
  \label{eq:NRJ_zhang}
  E(\phi) = \eta_1 \int_\Omega \left( \frac{\partial \phi}{\partial x}\omega_x + \frac{\partial \phi}{\partial y}\omega_y + \frac{\partial \phi}{\partial v} \right)^2 \, dxdy + \eta_2 \int_\Omega \left( \frac{\partial \phi}{\partial u} - \omega_u \right)^2 \, dxdy.
\end{equation}

\vspace{0.3cm}
\paragraph{Evolution equation}
~\par \vspace{0.3cm}
Given an initial \ls~function $\phi_0$, the functional \refeq{eq:NRJ_zhang} is minimized by iterating two steps, first using the current estimate of $\phi$ to estimate $c_1$, $c_2$ and solving for $\omega_x(u, v)$, $\omega_y(u, v)$, and $\omega_u(u, v)$, and then evolving $\phi$ by:
\begin{eqnarray}
  \label{eq:dphidt_zhang}
  \nonumber \frac{\partial \phi}{\partial t} = &\delta(\phi(\vecx{x})) \nabla_\phi F(I(\vecx{x}), \phi(\vecx{x})) + \lambda \delta(\phi(\vecx{x})) \text{div}\left(\frac{\nabla \phi(\vecx{x})}{\|\nabla \phi(\vecx{x})\|}\right) + 2\eta_2 \left( \dfrac{\partial^2 \phi}{\partial u^2} - \dfrac{\partial \omega_u}{\partial u} \right) \\
 & + 2\eta_1 \left( \dfrac{\partial^2 \phi}{\partial x^2}\omega_x^2 + \dfrac{\partial^2 \phi}{\partial y^2}\omega_y^2 + \dfrac{\partial^2 \phi}{\partial v^2} + 2\dfrac{\partial^2 \phi}{\partial x\partial y}\omega_x\omega_y + 2\dfrac{\partial^2 \phi}{\partial x\partial v}\omega_x + 2\dfrac{\partial^2 \phi}{\partial y\partial v}\omega_y \right)
\end{eqnarray}


\vspace{0.3cm}

\newpage
\section[Cremers and Soatto]{Cremers and Soatto 2005 \cite{Cremers2005, Cremers2003}}
\label{sec:motion-cremers}

\paragraph{Energy criterion}
~\par \vspace{0.3cm}
Let $\Omega \in \Rset^2$ denote the image plane and let $f: \Omega \times \Rset \rightarrow \Rset$ be a gray value image sequence. Denote the spatio-temporal image gradient of $f(\vecx{x}, t)$ by $\nabla_3 f = \left( \frac{\partial f}{\partial x_1}, \frac{\partial f}{\partial x_2}, \frac{\partial f}{\partial t} \right)^t$ and let $ v : \Omega \rightarrow \Rset^3, \, v(\vecx{x}) = ( u(\vecx{x}), w(\vecx{x}), 1)^t$ be the velocity vector at a point $\vecx{x}$ in homogeneous coordinates.

With these definitions, the problem of motion estimation now consists in maximizing the conditional probability
\begin{equation}
  \label{eq:Proba_v-gradf_cremers2005}
  P(v | \nabla_3 f) = \frac{P(\nabla_3 f | v)P(v)}{P(\nabla_3 f)}
\end{equation}
with respect to the motion field $v$.

Except for locations where the spatio-temporal gradient vanishes, the \emph{optic flow constraint} 
\begin{equation}
  \label{eq:optic_flow}
  \frac{df}{dt} = \frac{\partial f}{\partial t} + \frac{\partial f}{\partial x_1} \frac{dx_1}{dt} + \frac{\partial f}{\partial x_2} \frac{dx_2}{dt} = v^t \nabla_3f = 0
\end{equation}
states that the homogeneous velocity vector must be orthogonal to the spatio-temporal image gradient. Therefore a measure of this orthogonality is used as a conditional probability on the spatio-temporal image gradient:
\begin{equation}
  \label{eq:Proba_gradf-v_cremers2005}
  P(\nabla_3 f | v) \varpropto \exp \left( - \frac{( v(\vecx{x})^t \nabla_3 f(\vecx{x}) )^2}{ |v(\vecx{x})|^2 |\nabla_3 f(\vecx{x})|^2} \right)
\end{equation}

The velocity field $v$ is discretized by a set of disjoint regions $\Omega_i \subset \Omega$ with velocity $v_i$: $v(\vecx{x}) = {v_i,\text{ if }\vecx{x} \in \Omega_i}$. The prior probability on the velocity field is assumed to only depend on the length $L(C)$ of the boundary $C$ separating these regions:
\begin{equation}
  \label{eq:Proba_v_cremers2005}
  P(v) \varpropto \exp \left( - \mu L(C) \right)
\end{equation}

With the above assumptions, the first term in the numerator of \refeq{eq:Proba_v-gradf_cremers2005} can be written as:
\begin{equation}
  \label{eq:Proba_gradf-v2_cremers2005}
  P(\nabla_3 f | v) = \prod_{x \in \Omega} P(\nabla_3 f(\vecx{x}) | v(\vecx{x}))^h = \prod_{i=1}^n { \prod_{x \in \Omega_i} { P(\nabla_3 f(\vecx{x}) | v_i)^h } },
\end{equation}
where $h = d\vecx{x}$ denotes the pixel size of the discretization of $\Omega$.

Maximizing the conditional probability \refeq{eq:Proba_v-gradf_cremers2005} with respect to the velocity field $v$ is equivalent to minimize the negative logarithm of this expression, which is given by the energy functional:
\begin{equation}
  \label{eq:NRJ_C_cremers2005}
  E(C,\lbrace v_i \rbrace) = \sum_{i=1}^n {\int_{\Omega_i} \frac{( v(\vecx{x})^t \nabla_3 f(\vecx{x}) )^2}{ |v(\vecx{x})|^2 |\nabla_3 f(\vecx{x})|^2} \, d\vecx{x} + \mu L(C) }.
\end{equation}
Using the \ls~function $\phi$ (for a two-phase segmentation), eq.\refeq{eq:NRJ_C_cremers2005} can be rewritten:
\begin{equation}
  \label{eq:NRJ_phi_cremers2005}
  E(\phi,v_1, v_2) = \int_\Omega \frac{( v_1(\vecx{x})^t \nabla_3 f(\vecx{x}) )^2}{ |v_1(\vecx{x})|^2 |\nabla_3 f(\vecx{x})|^2} H(\phi(\vecx{x})) \, d\vecx{x} + \int_\Omega \frac{( v_2(\vecx{x})^t \nabla_3 f(\vecx{x}) )^2}{ |v_2(\vecx{x})|^2 |\nabla_3 f(\vecx{x})|^2} (1-H(\phi(\vecx{x}))) \, d\vecx{x} + \mu L(C).
\end{equation}


\paragraph{Velocity field expression}
~\par \vspace{0.3cm}
In order to cope with complex motion regions, piecewise parametric motion can be used. The velocity on the domain $\Omega_i$ is allowed to vary according to a model of the form:
\begin{equation}
  \label{eq:v-model_cremers2005}
  v_i(\vecx{x}) = M(\vecx{x})p_i,
\end{equation}
where $M$ is a matrix depending only on space and time and $p_i$ is the parameter vector associated with each region.

Inserting model \refeq{eq:v-model_cremers2005} into the \emph{optic flow constraint} \refeq{eq:optic_flow} gives a relation which states that the vector $M^t \nabla_3 f$ must either vanish or be orthogonal to the vector $p_i$. The energy functional can thus be rewritten as:
\begin{equation}
  \label{eq:NRJ2_phi_cremers2005}
  E(\phi,p_1, p_2) = \int_\Omega \frac{p_1^t T(\vecx{x})p_1}{|p_1|^2} H(\phi(\vecx{x})) \, d\vecx{x} + \int_\Omega \frac{( p_2^t T(\vecx{x})p_2 )^2}{|p_2|^2} (1-H(\phi(\vecx{x}))) \, d\vecx{x} + \mu L(C),
\end{equation}
where $T(\vecx{x}) = \dfrac{ \nabla_3 f^t(\vecx{x}) M(\vecx{x}) M^t(\vecx{x}) \nabla_3 f(\vecx{x}) }{ |M^t(\vecx{x}) \nabla_3 f(\vecx{x})|^2 }$ (\todo{$T(\vecx{x}) = \dfrac{ M(\vecx{x}) \nabla_3 f^t(\vecx{x}) \nabla_3 f(\vecx{x}) M^t(\vecx{x}) }{ |M^t(\vecx{x}) \nabla_3 f(\vecx{x})|^2 }$??}).


\paragraph{Evolution equation}
~\par \vspace{0.3cm}
For fixed motion vectors, the gradient descent on the functional \refeq{eq:NRJ_phi_cremers2005} for the \ls~function $\phi$ is given by:
\begin{equation}
  \label{eq:dphidt_cremers2005}
  \frac{\partial \phi}{\partial t} = \delta(\phi(\vecx{x})) \left[ \mu \cdot \text{div} \left( \frac{\nabla \phi}{\| \nabla \phi \|} \right) + e_2 - e_1 \right]
\end{equation}
with the energy densities $e_i$ given by:
\begin{equation}
  \label{eq:ei_cremers2005}
  e_i = \frac{p_i^t T(\vecx{x}) p_i}{p_i^t p_i} = \frac{p_i^t \nabla_3 f^t(\vecx{x}) M(\vecx{x}) M^t(\vecx{x}) \nabla_3 f(\vecx{x}) p_i}{|p_i|^2 |M^t(\vecx{x}) \nabla_3 f(\vecx{x})|^2}
\end{equation}


For fixed $\phi$, minimization of the functional \refeq{eq:NRJ_phi_cremers2005} with respect to the motion vectors $p_1$ and $p_2$ results in the eigenvalue problem:
\begin{equation}
  \label{eq:dpdt_cremers2005}
  p_i = \underset{p}{argmin} \left( \frac{p^t T_i(\vecx{x}) p}{p^t p} \right),
\end{equation}
for the matrices
\begin{eqnarray}
  \label{eq:Ti_cremers2005}
  T_1 = \int_\Omega T(\vecx{x})H(\phi((\vecx{x})) \, d\vecx{x} \text{ \hspace*{0.3cm} and}\\
  T_2 = \int_\Omega T(\vecx{x})(1-H(\phi((\vecx{x}))) \, d\vecx{x}
\end{eqnarray}

The solution of \refeq{eq:dpdt_cremers2005} is given by the eigenvectors corresponding to the smallest eigenvalues of $T_1$ and $T_2$, normalized such that its last component is 1.


\newpage
\cite{Papin2000}

% -- End of the declaration of each part
%___________________________________________________________
%

\newpage

\pagestyle{fancy}
\fancyhead{} % Headers
\fancyhead[LE,RO]{\nouppercase\leftmark}

\bibliography{src/bibliographie}
\bibliographystyle{IEEEbib}
\addcontentsline{toc}{chapter}{Bibliography}

\end{document}